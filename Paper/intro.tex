% !TeX root = ../AdaptiveSeedingSTOC.tex
With the rising public use of the Internet in modern times, companies 
such as Google or Facebook are gathering increasingly massive amounts 
of data. These companies often want to leverage their huge databases to 
learn and predict hidden information. For example, Netflix utilizes user 
ratings to improve their recommendation system, which finds new and 
interesting movies for their users to watch. In order to properly 
process these mountains of information, researchers have invested
considerable into the development of distributed algorithms for 
various machine learning tasks. For example, the Noisy Matrix Factorization 
problem is at the core of many online recommendation systems, and is
also useful in concisely representing data. Recently, a strongly parallel 
algorithm for this problem called Divide-Factor-Combine was developed 
at UC Berkeley[CITE]. 

These machine learning algorithms often take in many parameters, and
to achieve the best results, the input parameters must be optimized for
each specific application. In addition, the use of parallelism and the
nature of convergent algorithms creates a tradeoff between time, accuracy,
and the number of processors or machines tasked with the computation. 
On the one hand, too much parallelism can cause the algorithm to lose 
error guarantees and perform inadequately. On the other hand, too 
little parallelism can make the computation take an unacceptably long 
amount of time. On top of that, monetary budgets may restrict the number
of machines or processors one is willing to dedicate to the task, limiting
the capacity for parallelism. The objectives of minimizing both the error 
and time of the computation within a monetary budget are obviously 
in conflict.

Further, different users may have different requirements over these 
domains. In the case of an online recommendation system, as more 
customers buy or rate products, it is necessary to rerun the machine 
learning task in order to ensure relevant and current recommendations. 
Here the user may have some time budget describing how long he can wait 
for the machine learning algorithm to run, and would want to minimize 
the prediction error of the algorithm subject to this budget. On the other
hand, a genome scientist who has collected partial genetic data 
across many organisms may care more about maximizing the accuracy of the
phenotype prediction, but have a much more relaxed time budget. 

Either extensive empirical tests or an operator who is extremely 
familiar with both the dataset and the algorithm is required in order 
to achieve optimal results for a particular problem instance. In addition,
optimized parallelization choices might meet time budgets on one particular
architecture but fail on another, as communication times and machine
speeds can vary drastically. For these reasons, it is very desirable 
to have an automated process that can pick settings that meet budgets
reliably, and solve the problem optimally within budget constraints; such
an automated process could be architecture-independent, and adapt 
as the distribution of input data changes over time. 

\subsection{Previous Work}
There has been a large amount of work on automatic parameter choice, 
especially for regularization parameters in various regression algorithms 
[CITE,CITE,CITE]. However, many of these results depend on assuming some 
underlying structure in the data, and exploiting that in order to provide 
analytic techniques that compute the optimal or near-optimal choice of 
parameter. Many of these automatic tuning algorithms are also extremely 
application-specific. 

To our knowledge, there is no general framework for automating parameter 
choice that can be applied independent of algorithm or task. In this work 
we propose a system to optimize parameters that depends only on the input 
and output of the algorithm. Our system does not assume any underlying 
data model, and only uses its own history to estimate optimal parameters. 
This approach may give worse results than the model approach if the model 
is carefully chosen, but this is not always possible, and it does not
allow for changes to the data over time.

\subsection{Our Contribution}
In this work we propose an optimizer for choosing parameter settings for 
distributed machine learning algorithms, in particular for distributed 
matrix computations. Given input data and user-specified budgets, our 
optimizer automatically chooses algorithm parameters to minimize either
time, error, or monetary expenditure while not going over-budget in any 
of these realms. The optimizer maintains a database of the parameter 
choices and empirical performance of previous jobs, and uses this stored 
information to choose optimal parameters for new incoming jobs.

Our optimizer has done quite well in tests, consistently performing as 
well as the best possible manual choice of parameters. Our preliminary 
implementation currently optimizes the Divide-Factor-Combine algorithm 
for Noisy Matrix Factorization. Our tests were preformed on both synthetic 
and real-world data. Our synthetic data was in the form of Gaussian 
random matrices, and our real-world data was the Movielens10M 
dataset[CITE]. We have designed out optimizer to be general and easily 
extensible, and adding additional machine learning algorithms to our 
optimizer framework is the subject of ongoing work.
